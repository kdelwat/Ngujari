\chapter{Nominal Morphology}

\section{Gender}\index{Gender}

Ngujari has four genders: child, adult, elder (grouped together as animate), and
inanimate. Gender is assigned semantically and changes the morphosyntactic
alignment of the sentence as well as posessives.

The animate gender is given to people, animals, and Dreamtime figures. For
example, \textit{Yawirra}, the concept of the Land, is considered animate. The
inanimate gender applies to all other nouns.

Within the animate there are three genders, each representing a different stage
in life. This distinction is important in areas such as pronouns, but not in
others, like verbal inflection. An animate noun is assigned to a stage based on
their social position. Those who are yet to undergo the adulthood ceremony
(those under roughly 14 in the case of females and 16 in the case of males) are
assigned the child gender, while those who have become elders receive the elder
gender. All other ages are grouped into the adult gender.

\section{Cases}\index{Cases}

Ngujari has eight nominal cases, with three indicating the morphosyntactic
alignment and five others. Cases are indicated by single-syllable suffixes, as
indicated in the following table.

\begin{table}[h]
\centering
\begin{tabular}{lcc}
\textbf{case} & \textbf{abbreviation} & \textbf{suffix}\\
ergative & \textsc{erg} & -\\
nominative & \textsc{nom} & -wa\\
accusative & \textsc{abs} & -rru\\
instrumental & \textsc{ins} & -ma\\
comitative & \textsc{com} & -yii\\
orientative & \textsc{ori} & -rni\\
revertive & \textsc{rev} & -nga\\
locative & \textsc{loc} & -ru\\
\end{tabular}
\caption{Case Suffixes}
\end{table}

For more details on the three alignment cases, see \ref{sec:alignment} (pg.
\pageref{sec:alignment}). The remaining five cases operate as follows:

\begin{description}
\item[instrumental] The instrumental case is used when discussing a *means*,
  roughly equivalent to the English ``by means of''. For example, when speaking
  of killing a fish using a spear, a Ngujari speaker will place ``spear'' in the
  \textsc{INS} case.
\item[comitative] The comitative case is equivalent to ``in the presence of'',
  or ``with'', and specifies that the noun was present at the moment spoken of.
\item[orientative] The orientative case is used to specify that something is
  facing towards the noun. It is often used with the meaning of ``heading
  towards''.

  aux 2s-ERG camp-ORI togo-an-2nd.

  You are heading towards the camp.

\item[revertive] The revertive case is used to specify that something is
    oriented away from the noun. It can be used with the meaning of ``heading
    away from''.

  aux 3pl-an-NOM 3s-an-REV togo-an-3rd.

  They are heading away from her.

  It can also be used in asserting falsehood.

  aux-remote 3s-an-ERG knowledge-NOM valence1->2 tolook-an-3rd.

  He used to look away from knowledge / he used to be incorrect.

\item[locative] The locative case is used to specify a location, and can take
  the place of a preposition such as ``in'' or ``at''. This means that ``she is
  at the house'' is equivalent to ``she is [house] (\textsc{LOC})''. The
  locative suffix *-ru* becomes a long u if placed after a word ending in a
  short u.

\end{description}

An example of the use of these cases is found in the following table, which
shows the declensions of the noun \textit{naju}, or ``rock''.

\begin{table}[h]
\centering
\begin{tabular}{lll}
\textbf{case} & \textbf{word} & \textbf{meaning}\\
ergative & naju & -\\
nominative & najuwa & -\\
accusative & najurru & -\\
instrumental & najuma & ``using the rock''\\
comitative & najuyii & ``in the presence of the rock''\\
orientative & najurni & ``oriented towards the rock''\\
revertive & najunga & ``orientated away from the rock''\\
locative & najuu & ``at the rock''\\
\end{tabular}
\caption{Examples of Nominal Case Declensions}
\end{table}