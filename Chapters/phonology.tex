\chapter{Phonology}

\section{Phonetic Inventory}\index{Phonetic Inventory}

In terms of phonology, Ngujari has a rich consonantal inventory featuring a
series of coronal consonants (both laminal and apical), as well as multiple
rhotics. The following table shows the consonants and their orthographic
representation in italics (if different from the IPA).

\begin{table}[h]
\centering
\begin{tabular}{rcccccc}
  & \textbf{bilabial} & \textbf{alveolar} & \textbf{post-alveolar} & \textbf{retroflex} & \textbf{palatal} & \textbf{velar}\\
  \textbf{plosive} & \textipa{p} & \textipa{\|]{t}}\textit{(t)} & & \textipa{\|{]}{\:t}}\textit{(rt)} & & \textipa{k}, \textipa{g}\\
  \textbf{nasal} & \textipa{m} & \textipa{\|]{n}}\textit{(n)} & \textipa{\textsubsquare{n}}\textit{(nn)} & \textipa{\|{]}{\:n}}\textit{(rn)} & & \textipa{N}\textit{(ng)}\\
  \textbf{trill} & & \textipa{\|]{r}}\textit{(rr)} & & & &\\
  \textbf{tap} & & \textipa{\|]R}\textit{(rr)} & & & &\\
  \textbf{fricative} & & & \textipa{Z}\textit{(j)} & & &\\
  \textbf{approximant} & \textipa{w} & & & \textipa{\:R}\textit{(r)} & \textipa{j}\textit{(y)} &\\
  \textbf{lateral approximant} & & \textipa{\|]{l}}\textit{(l)} & & \textipa{\|]{\:l}}\textit{(rl)} & &\\
\end{tabular}
\caption{Consonantal Inventory}
\end{table}

The vowel palette is very restricted, limited to just a, i, and u, as well as
their lengthened versions, represented orthographically by repeating the letter.

\begin{table}[h]
\centering
\begin{tabular}{rcc}
& \textbf{front} & \textbf{back}\\
\textbf{high} & \textipa{i}, \textipa{i:} & \textipa{u}, \textipa{u:}\\
\textbf{low} & \textipa{a}, \textipa{a:} &\\
\end{tabular}
\caption{Vowel Inventory}
\end{table}

\section{Phonotactics}\index{Phonotactics}

Some phonotactic rules apply:

\begin{itemize}
\item Syllables take the form C\textsubscript{1}V\textsubscript{1} (C\textsubscript{2}).
\item A word is usually 2--4 syllables plus one or more single-syllable suffixes.
\item Words may not begin with a liquid or retroflex consonant.
\item Stress always falls on the first syllable of each word.
\end{itemize}

\subsection{Historical Sound Changes}\index{Historical Sound Changes}

Ngujari differs phonologically from Proto-Pama-Nyungan only slightly. The
following is a list of sound changes that have occured:

\begin{itemize}
\item Apicalised post-alveolar plosive (\textipa{\textsubsquare{t}}) becomes
  voiced post-alveolar fricative (\textipa{Z}).
\item Apicalised alveolar trill (\textipa{\|]r}) becomes apicalised alveolar tap
  (\textipa{\|]R}) immediately following regular vowels.
\item Unvoiced velar plosive (\textipa{k}) voices to \textipa{g} following
  \textipa{u} or \textipa{u:}.
\item Retroflex approximant (\textipa{\:R}) disappears between identical regular
  vowels, forming one lengthened vowel.
\item Apicalised alveolar lateral approximant (\textipa{\|]l}) disappears from
  the end of words.
\end{itemize}

A major difference occurs in the case of lengthened vowels, which can
differentiate words in all positions, rather than just the first syllable as in
the protolanguage.
