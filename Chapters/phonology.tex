\chapter{Phonology}

\section{Phonetic Inventory}\index{Phonetic Inventory}

\subsection{Consonants}\index{Consonants}

In terms of phonology, Ngujari has a rich consonantal inventory featuring a
large series of coronal consonants (both laminal and apical) and multiple
rhotics. The following table shows the consonants and their orthographic
representation in italics (if different from the IPA).

\begin{table}[h]
\centering
\begin{tabular}{rcccccc}
  & \textbf{bilabial} & \textbf{alveolar} & \textbf{post-alveolar} & \textbf{retroflex} & \textbf{palatal} & \textbf{velar}\\
  \textbf{plosive} & \textipa{p} & \textipa{\|]{t}}\textit{(t)} & & \textipa{\|{]}{\textrtailt}}\textit{(rt)} & & \textipa{k}, \textipa{g}\\
  \textbf{nasal} & \textipa{m} & \textipa{\|]{n}}\textit{(n)} & \textipa{\textsubsquare{n}}\textit{(nn)} & \textipa{\|{]}{\textrtailn}}\textit{(rn)} & & \textipa{N}\textit{(ng)}\\
  \textbf{trill} & & \textipa{\|]{r}}\textit{(rr)} & & & &\\
  \textbf{tap} & & \textipa{\|]R}\textit{(rr)} & & & &\\
  \textbf{fricative} & & & \textipa{Z}\textit{(j)} & & &\\
  \textbf{approximant} & \textipa{w} & & & \textipa{\textturnrrtail}\textit{(r)} & \textipa{j}\textit{(y)} &\\
  \textbf{lateral approximant} & & \textipa{\|]{l}}\textit{(l)} & & \textipa{\|]{\textrtaill}}\textit{(rl)} & &\\
\end{tabular}
\caption{Consonantal Inventory}
\end{table}

\subsection{Vowels}\index{Vowels}

The vowel palette is very restricted, limited to just a, i, and u, as well as
their lengthened versions. The long vowels are contrastive in all locations.
These phonemes are found in the following table.

\begin{table}[h]
\centering
\begin{tabular}{rcc}
& \textbf{front} & \textbf{back}\\
\textbf{high} & \textipa{i}, \textipa{i:} & \textipa{u}, \textipa{u:}\\
\textbf{low} & \textipa{a}, \textipa{a:} &\\
\end{tabular}
\caption{Vowel Inventory}
\end{table}

Orthographically, the short vowels are expressed according to their IPA
representation. Long vowels are simply the short vowel doubled.

The front vowels (i and a) are phonetically tense. Both have a nasalised
allophone.

The back vowel u is divided allophonetically into two sounds: the default u, and
the somewhat centralised \textipa{\"u} which tends towards the \textipa{U} sound
and is accordingly more lax than the default.

\section{Syllables}\index{Syllables}

\begin{itemize}
\item Syllables take the form C\textsubscript{1}V\textsubscript{1} (C\textsubscript{2}).
\item A word is usually 2--4 syllables plus one or more single-syllable suffixes.
\item Stress always falls on the first syllable of each word.
\end{itemize}

\section{Phonotactics}\index{Phonotactics}

Some allophonetic variation occurs during pronounciation.

\subsection{Consonant Clusters}\index{Consonant Clusters}

Consonant clusters only ever have two consonants, owing to the syllable
structure of the language, and accordingly are easy to classify. Four general
rules apply to the legal clusters in the language:

1. Retroflex consonants can not appear in a cluster.
2. Two coronal consonants can not appear together in a cluster.
3. Two plosives can not appear together in a cluster.
4. An approximant or fricative can not be the first consonant in the cluster.

plosives unreleased before nasals



\subsection{Vowels}

The u phoneme becomes centralised following bilabial consonants p, m, and w.

The i and a phonemes are nasalised before alveolar and post-alveolar nasals.

i->\textipa{\~i}
a->\textipa{\~a}

\subsection{Rhotics}

\phonl{\textipa{\|]r}}{\textipa{\|]R}}{ShortV}

Retroflex approximant (\textipa{\textturnrrtail}) disappears between identical
regular vowels, forming one lengthened vowel.

\subsection{Voicing}

The voicing process is relatively new to the language, and accordingly not much
variation is present. Generally, plosives are becoming initially voiced.
However, in practice the voiced plosive g is the only new voiced consonant
sufficiently formed to be included as an individual phoneme; the rest are in the
process of undergoing the differentiation. In the case of the t phoneme, only
the alveolar form undergoes voicing, as the retroflex cannot begin a word.

k->g
p->\textipa{\v*p}
alveolar t->\textipa{\t*p}

\subsection{Historical Sound Changes}\index{Historical Sound Changes}

Ngujari differs phonologically from Proto-Pama-Nyungan only slightly. The
following is a list of sound changes that have occured:

\begin{itemize}
\item Apicalised post-alveolar plosive (\textipa{\textsubsquare{t}}) becomes
  voiced post-alveolar fricative (\textipa{Z}).
\item Apicalised alveolar trill (\textipa{\|]r}) becomes apicalised alveolar tap
  (\textipa{\|]R}) immediately following regular vowels.
\item Retroflex approximant (\textipa{\textturnrrtail}) disappears between
  identical regular vowels, forming one lengthened vowel.
\item Apicalised alveolar lateral approximant (\textipa{\|]l}) disappears from
  the end of words.
\end{itemize}

A major difference occurs in the case of lengthened vowels, which can
differentiate words in all positions, rather than just the first syllable as in
the protolanguage.
