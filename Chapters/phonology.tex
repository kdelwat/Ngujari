\chapter{Phonology}

\section{Phonetic Inventory}\index{Phonetic Inventory}

\subsection{Consonants}\index{Consonants}

In terms of phonology, Ngujari has a rich consonantal inventory featuring a
large series of coronal consonants (both laminal and apical) and multiple
rhotics. The following table shows the consonants and their orthographic
representation in italics (if different from the IPA).

\begin{sidewaystable}
\centering
\begin{tabular}{rcccccc}
  & \textbf{bilabial} & \textbf{alveolar} & \textbf{post-alveolar} & \textbf{retroflex} & \textbf{palatal} & \textbf{velar}\\
  \textbf{plosive} & \textipa{p} & \textipa{\|]{t}}\textit{(t)} & & \textipa{\|{]}{\textrtailt}}\textit{(rt)} & & \textipa{k}, \textipa{g}\\
  \textbf{nasal} & \textipa{m} & \textipa{\|]{n}}\textit{(n)} & \textipa{\textsubsquare{n}}\textit{(nn)} & \textipa{\|{]}{\textrtailn}}\textit{(rn)} & & \textipa{N}\textit{(ng)}\\
  \textbf{tap} & & \textipa{\|]R}\textit{(rr)} & & & &\\
  \textbf{fricative} & & & \textipa{Z}\textit{(j)} & & &\\
  \textbf{approximant} & \textipa{w} & & & \textipa{\textturnrrtail}\textit{(r)} & \textipa{j}\textit{(y)} &\\
  \textbf{lateral approximant} & & \textipa{\|]{l}}\textit{(l)} & & \textipa{\|]{\textrtaill}}\textit{(rl)} & &\\
\end{tabular}
\caption{Consonantal Inventory}
\end{sidewaystable}

\subsection{Vowels}\index{Vowels}

The vowel palette is very restricted, limited to just a, i, and u, as well as
their lengthened versions. The long vowels are contrastive in all locations.
These phonemes are found in the following table.

\begin{table}[h]
\centering
\begin{tabular}{rcc}
& \textbf{front} & \textbf{back}\\
\textbf{high} & \textipa{i}, \textipa{i:} & \textipa{u}, \textipa{u:}\\
\textbf{low} & \textipa{a}, \textipa{a:} &\\
\end{tabular}
\caption{Vowel Inventory}
\end{table}

Orthographically, the short vowels are expressed according to their IPA
representation. Long vowels are simply the short vowel doubled.

The front vowels (i and a) are phonetically tense. Both have a nasalised
allophone.

The back vowel u is divided allophonetically into two sounds: the default u, and
the somewhat centralised \textipa{\"u} which tends towards the \textipa{U} sound
and is accordingly more lax than the default.

\section{Phonotactics}\index{Phonotactics}

\subsection{Syllables and Morae}\index{Syllables}

The structure of Ngujari words is simple, with syllables taking the form CV: one
consonant is followed by one vowel. A root word is usually between two and four
syllables long, plus any affixes which tend to be single-syllable. In addition,
words can be broken into \textit{morae}. A syllable containing a short vowel is
worth one mora, but those containing long vowels are worth two. This distinction
becomes important when dealing with prosody in \autoref{prosody}.


\subsection{Vowels}

The \textit{u} phoneme becomes centralised following some bilabial consonants p, m, and w.

%\begin{quote}
%\phonc{u}{\textipa{\"u }}{\oneof{p \phold\\ m \phold\\ w \phold}}
%\end{quote}

The i and a phonemes are nasalised before alveolar and post-alveolar nasals.

%\begin{quote}
%\begin{multicols}{2}
%\phonc{i}{\textipa{\~i}}{\oneof{\phold \textipa{\|]{n}}\\ \phold
    %\textipa{\textsubsquare{n}}}}
%
%\phonc{a}{\textipa{\~a}}{\oneof{\phold \textipa{\|]{n}}\\ \phold
    %\textipa{\textsubsquare{n}}}}
%\end{multicols}
%\end{quote}

\subsection{Consonants}

\subsubsection{Rhotics}

The retroflex approximant \textipa{\textturnrrtail} disappears between identical
regular vowels, forming one lengthened vowel.

%\begin{quote}
%%\phonc{\textipa{\textturnrrtail}}{\o}{\oneof{a \phold a\\ u \phold u\\ i \phold i}}
%\end{quote}
%
\subsubsection{Voicing}

The voicing process is relatively new to the language, and accordingly not much
variation is present. Generally, plosives are becoming initially voiced.
However, in practice the voiced plosive g is the only new voiced consonant
sufficiently formed to be included as an individual phoneme; the rest are in the
process of undergoing the differentiation. In the case of the \textipa{\|]{t}}
phoneme, only the alveolar form undergoes voicing, as the retroflex cannot begin
a word.

%\begin{quote}
%\begin{multicols}{3}
%\phonl{k}{g}{\^{}}
%
%\phonl{p}{\textipa{\v*p}}{\^{}}
%
%\phonl{\textipa{\|]{t}}}{\textipa{\t*p}}{\^{}}\footnote{The phoneme remains
  %apical, but this cannot be expressed in IPA.}
%\end{multicols}
%\end{quote}

\subsection{Historical Sound Changes}\index{Historical Sound Changes}

Ngujari differs phonologically from Proto-Pama-Nyungan only slightly. The
following is a list of sound changes that have occured:

\begin{itemize}
\item Apicalised post-alveolar plosive (\textipa{\textsubsquare{t}}) becomes
  voiced post-alveolar fricative (\textipa{Z}).
\item Apicalised alveolar trill (\textipa{\|]r}) becomes apicalised alveolar tap
  (\textipa{\|]R}).
\item Retroflex approximant (\textipa{\textturnrrtail}) disappears between
  identical regular vowels, forming one lengthened vowel.
\item Apicalised alveolar lateral approximant (\textipa{\|]l}) disappears from
  the end of words.
\end{itemize}

A major difference occurs in the case of lengthened vowels, which can
differentiate words in all positions, rather than just the first syllable as in
the protolanguage.

\section{Prosody}\index{Prosody}\label{prosody}

Ngujari has a rich prosodic system incorporating stress, intonation, and tempo.
Stress is dealt with here, but intonation and tempo are left to Part 2 in the
discussion on pragmatics.

\subsection{Stress}
Stress follows a simple process. The primary stress is placed on the second mora
of the word. If that mora is part of the first syllable (i.e. the first syllable
has a long vowel rendering it bimoraic), the first syllable is stressed.
Secondary stress is then placed on morae at even intervals, on the 4th, 6th,
etc. However, if the secondary stress would fall on the second mora of a
bimoraic syllable, it is skipped.
