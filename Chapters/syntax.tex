\chapter{Syntax}

\section{Alignment}\index{Morphosyntactic Alignment}\label{alignment}

The alignment of Ngujari depends on whether the noun in question is an animate
pronoun or not. For clauses with exclusively animate pronouns, the alignment is
nominative-accusative, but otherwise it is ergative-nominative (i.e. the
transitive patient and intransitive object are marked nominative and the
transitive agent is marked ergative). This system applies only to intransitive
and transitive verbs. For higher valencies, formed through \ref{valencemod}, the
extra arguments are assigned cases semantically.

\section{Verb Phrases}\index{Verb Phrases}

\begin{definition}[Verb Phrase]
~\\
\textsc{vp = aux [neg] np(s) [adv(s)] [val] v}
\end{definition}

Verb phrases can be as simple as a single avalent verb, such as in ``it's raining'', or as complex as a tetravalent causative.

In the prototypical verb clause, the following rules govern word order:

\begin{enumerate}
\item The verb's auxiliary appears at the beginning.
\item The verb itself appears at the end.
\item Valence modifiers appear immediately before the verb.
\end{enumerate}

The following examples illustrate basic verb phrases:

\begin{sentence}
\shortex{5}
{\bfseries k-i & \bfseries wa-j & \bfseries kurru-l & \bfseries ji &
  \bfseries wurr-u-\o.}
{strike.\textsc{aux-pres} & 1\textsc{s-nom} & 2\textsc{s-ACC} & \textsc{0.val.2} & electrically.storm-\textsc{an}-\textsc{1st}}
{\textit{I strike you.}}

\shortex{4}
{\bfseries wann-uma & \bfseries maaju-maaju-j & \bfseries ka & \bfseries jinn-u-m.}
{see.\textsc{aux-pst} & kangaroo-\textsc{pl-nom} & \textsc{2.val.1} & eat-\textsc{an-3rd}}
{\textit{The kangaroos ate/were eating.}}
\end{sentence}

Noun phrases tend to appear in order of importance to the statement as judged by
the speaker.

\section{Noun Phrases}\index{Noun Phrases}

\begin{definition}[Verb Phrase]
~\\
\textsc{np = [adj(s)-attr] n [rel(s)]}
\end{definition}

A noun phrase consists of one noun, declined by case, and any number of
adjectives and relative clauses. The noun tends to be placed first, followed by
adjectives, although this can be inverted or even mixed according to pragmatic
considerations. However, relative clauses always succeed the noun and
adjectives.

\begin{sentence}

\shortex{3}
{\textbf{birru-\o} &\textbf{birruku} &\textbf{miinna} }
{sea-\textsc{ERG} &blue &big}
{\textit{vast blue sea}}

\shortex{3}
{\textbf{kanaama} &\textbf{yirlirna-j} &\textbf{gu} }
{woven &basket-\textsc{NOM} &small}
{\textit{small woven basket}}

\end{sentence}

\section{Relative Clauses}\index{Relative Clauses}\label{relativeclauses}

\begin{definition}[Relative Clause]
~\\
\textsc{vp = aux [neg] np(s) [adv(s)] [val] v}\\
$\Rightarrow$ \textsc{rc = aux [neg] v [val] [adv(s)] np(s)}
\end{definition}

Relative clauses are \textit{adjoined} to the noun phrase. The clause undergoes
a transformation from the standard verb phrase by moving the verb to the
position immediately following the auxiliary. The valency modifier is free to be
placed anywhere among the remaining noun phrases and adverbs, but typically
follows the verb.

If the head noun is a patient of the relative clause, the verb of the relative
clause has its valence reduced by one.

\begin{sentence}
\shortex{5}
{\textbf{wiingu-\o} &\textbf{k-a} &\textbf{pirr-u-\o} &\textbf{ka} &\textbf{wa-j}}
{man-\textsc{erg} &\textsc{aux}-\textsc{pst} &see-\textsc{an}-1\textsc{st} &2.\textsc{val}.1 &1s-\textsc{nom}}
{\textit{the man that I saw}}
\end{sentence}

If the head noun is the agent, a pronoun is used inside the relative clause to
refer back to it.

\begin{sentence}
\shortex{6}
{\textbf{j-a} &\textbf{Wuurna-\o} &\textbf{nn-uuki-ti} &\textbf{yann-u-mi} &\textbf{nna-j} &\textbf{jurlu-l}}
{say.\textsc{aux}-\textsc{pst} &Wuurna-\textsc{erg} &\textsc{aux}-\textsc{fut}-\textsc{dub} &catch-\textsc{an}-3\textsc{rd} &3s-\textsc{nom} &turtle-\textsc{acc}}
{}\\
\shortex{3}
{\textbf{wa-j} &\textbf{ka} &\textbf{naj-u-m} }
{1s-\textsc{nom} &3.\textsc{val}.2 &say-\textsc{an}-3\textsc{rd} }
{\textit{Wuurna, who might catch a turtle, spoke to me.}}

\end{sentence}

\subsection{Adverbial Phrases}\index{Adverbial Phrases}
\label{advsyntax}

Temporal adverbs, which specify the time an action takes place, tend to appear
following the noun.

\begin{sentence}
\shortex{6}
{\textbf{k-a} &\textbf{jana-\o} &\textbf{jari-rn} &\textbf{wiirr-uu-\o} &\textbf{yuurli-rna} &\textbf{ma} }
{go.\textsc{aux}-\textsc{pst} &1s.\textsc{ch}-\textsc{erg} &beach-\textsc{loc} &go-\textsc{ch}-1\textsc{st} &day-\textsc{rev} &one }
{\textit{Yesterday, I [a child] went to the beach.}}
\end{sentence}

Manner adverbs, which specify the manner in which the action was conducted,
usually appear directly before the noun.

\begin{sentence}
\shortex{5}
{\textbf{nn-uuki-yii} &\textbf{waya-\o} &\textbf{pirwa-pirwa-j} &\textbf{garrna} &\textbf{gann-u-\o} }
{pickup.\textsc{aux}-\textsc{fut}-\textsc{wimp} &1\textsc{pl}-\textsc{erg} &clothing-\textsc{pl}-\textsc{nom} &quickly &pickup-\textsc{an}-1\textsc{st} }
{\textit{We should pick up the clothes quickly.}}
\end{sentence}

However, both can occupy different positions inside the verb phrase if the
speaker desires it.

\section{Predicates}\index{Predicates}

There are three cases for predicates: adjectival, nominal, and locational.

In an adjectival predicative phrase a verb is not normally required. The noun is
assigned the same tense as it would be were it the argument to an intransitive
verb, while the adjective assumes its predicative inflection.

\begin{sentence}
\shortex{2}
{\textbf{puurna-j} &\textbf{birruku-ku} }
{sky-\textsc{nom} &blue-\textsc{an} }
{\textit{The sky is blue.}}
\end{sentence}

The adjectival predicative can be counter-intuitively used with a noun, by
placing the noun into the comitative case. This is usually employed when
describing a changeable state.

\begin{sentence}
\shortex{2}
{\textbf{murta-j} &\textbf{gurlu-yi} }
{berry-\textsc{nom} &freshness-\textsc{com} }
{\textit{The berry is fresh.}}
\end{sentence}

In a nominal predicative phrase, the verb ``to be'' is used. The predicate noun
is declined as verb's object.

\begin{sentence}
\shortex{4}
{\textbf{ngarr-i} &\textbf{wa-\o} &\textbf{gajangu-j} &\textbf{ngurr-u-\o} }
{be.\textsc{aux}-\textsc{pres} &1\textsc{s}-\textsc{erg} &teacher &be-\textsc{an}-1\textsc{st} }
{\textit{I am a teacher.}}
\end{sentence}

In a locational predicative phrase, the verb ``to be'' is still used, but the
predicate location is declined in the locative case.

\begin{sentence}
\shortex{4}
{\textbf{k-i} &\textbf{wurlki-\o} &\textbf{kirujunga-\o} &\textbf{ngurr-a-m} }
{be.\textsc{aux}-\textsc{pres} &village-\textsc{erg} &somewhere-\textsc{loc} &be-\textsc{inan}-3\textsc{rd} }
{\textit{The village is somewhere.}}
\end{sentence}

\section{Possession}\index{Possession}

\subsection{Alienable}

To indicate alienable possession (possession that is not permanent or subject to
change), the locative case is used in conjunction with the verb ``to be''. The
possessed noun appears in the locative case as the subject of the transitive
form of ``to be'', with the possessor appearing as the object in the usual case.

\begin{sentence}
\shortex{4}
{\textbf{ngarr-i} &\textbf{mulu-mulu-ka-rn} &\textbf{mungu-j} &\textbf{ngurr-a-m} }
{be.\textsc{aux}-\textsc{pres} &deadfish-\textsc{pl}-\textsc{dual}-\textsc{loc} &woman-\textsc{nom} &be-\textsc{inan}-3\textsc{rd} }
{\textit{The woman has two dead fish.}}
\end{sentence}

\subsection{Inalienable}

Inalienable possession (possession that is unequivocal) is indicated simply
through the use of the verb ``to have''.

\begin{sentence}
\shortex{5}
{\textbf{garr-aa-nga} &\textbf{ngungu-j} &\textbf{jarta-l} &\textbf{ka} &\textbf{gurr-u-\o} }
{have.\textsc{aux}-\textsc{fut}-\textsc{gno} &mob-\textsc{nom} &homeland-\textsc{acc} &3.\textsc{val}.2 &have-\textsc{an}-1\textsc{st} }
{\textit{Our mob will always have a homeland.}}
\end{sentence}

\subsection{Pronominal}

A noun phrase can be indicated as possessed through the use of a possessive
pronoun as an adjective.

\begin{sentence}
\shortex{6}
{\textbf{nn-uma} &\textbf{nnaa-\o} &\textbf{waju-j} &\textbf{yurni} &\textbf{nna-lu} &\textbf{giinn-u-m} }
{admire.\textsc{aux}-\textsc{pst} &3\textsc{pl}.\textsc{an}-\textsc{erg} &face-\textsc{nom} &beautiful &3\textsc{s}.\textsc{an}-\textsc{pos} &admire-\textsc{an}-3\textsc{rd} }
{\textit{They admired his beautiful face.}}
\end{sentence}

In Ngujari culture, an object can be owned by a mob as a whole. Only inanimate
objects may be possessed by a mob (with the exception of areas of land).
Possession is indicated by the particle \textit{tuu}, which appears before the
noun. To specify the possessing mob, the mob's name is placed immediately after
the particle. The regular name is used by members of the possessing mob, but the
honorific name is used for possessions of others. For example, the particle for
something owned by the Wujanga mob would be \textit{tuu-Wujanga} for a member or
\textit{tuu-Wujarra} for an outsider.

\begin{sentence}
\shortex{5}
{\textbf{nn-i-ju} &\textbf{waya-\o} &\textbf{tuu-Gurnu} &\textbf{jaku} &\textbf{nnalu-j} }
{protect.\textsc{aux}-\textsc{strimp} &1\textsc{pl}-\textsc{erg} &\textsc{pos}-\textsc{g}urnu &precious &land-\textsc{nom} }
{\textit{}}
\shortex{3}
{\textbf{muu-ma} &\textbf{naa} &\textbf{tinn-u-\o} }
{spirit-\textsc{inst} & 2.\textsc{val}.3 & protect-\textsc{an}-1\textsc{st} }
{\textit{We must protect our [the Gurnu mob's] precious land with vigour.}}
\end{sentence}

\section{Verbal Constructions}\index{Verbal Constructions}

\subsection{Interrogative}\index{Interrogative}

\subsubsection{Polar Questions}

Polar questions are syntactically the same as a factual statement, except they
are expressed with a rising tone at the beginning of the question.

\begin{sentence}
\shortex{4}
{\textbf{nn-uuki} &\textbf{kupa-kupa-\o} &\textbf{gaypa-gaypa-rn} &\textbf{narnn-u-m?} }
{\textglobrise fly.\textsc{aux}-\textsc{fut} &bird-\textsc{pl}-\textsc{erg} &mountain-\textsc{pl}-\textsc{loc} &fly-\textsc{an}-3\textsc{rd} }
{\textit{Will the birds fly to the mountains?}}
\end{sentence}

\subsubsection{Non-Polar Questions}

One way of forming a non-polar question is using an interrogative pronoun as a
verb's argument, with no syntactic change taking place.

\begin{sentence}
\shortex{6}
{\textbf{kiru} &\textbf{ngarr-i} &\textbf{wumpa-j} &\textbf{j-i} &\textbf{palyaj-a-m} &\textbf{nnu-\o} }
{where &be.\textsc{aux}-\textsc{prs} &path-\textsc{nom} &leadto.\textsc{aux}-\textsc{prs} &leadto-\textsc{inan}-3\textsc{rd} &3\textsc{s}.\textsc{inan}-\textsc{erg} }
{}
\shortex{2}
{\textbf{wurlki-j} &\textbf{ngurr-a-m} }
{village-\textsc{nom} &be-\textsc{inan}-3\textsc{rd} }
{\textit{Where is the path that leads to the village?}}
\end{sentence}

To question a certain word in a statement, the particle *yuu* can be placed
before the word.

\begin{sentence}
\shortex{5}
{\textbf{k-aa} &\textbf{yuu-nnara-\o} &\textbf{nurtwu-j} &\textbf{panwa-rnu} &\textbf{mirr-uu-m} }
{bring.\textsc{aux}-\textsc{fut} &\textsc{int}-3\textsc{dual}.\textsc{an}-\textsc{erg} &food-\textsc{nom} &fire-\textsc{ori} &bring-\textsc{ch}-3\textsc{rd} }
{\textit{Will \textbf{those two children} bring the food to the fire?}}

\shortex{5}
{\textbf{nn-i} &\textbf{wa-\o} &\textbf{yuu-gurlurni} &\textbf{parnti-j} &\textbf{jinn-u-mi} }
{eat.\textsc{aux}-\textsc{pres} &3\textsc{s}-\textsc{erg} &\textsc{int}-fresh &kangaroomeat-\textsc{nom} &eat-\textsc{an}-3\textsc{rd} }
{\textit{Is he eating \textbf{fresh} kangaroo meat?}}
\end{sentence}

\subsection{Comparative}\index{Comparative Phrases}

Ngujari contains locational-type comparatives. This means that the
\textit{standard} noun, or the noun to be judged against, is marked in the
revertive case. Comparatives do not use a verb, and are always positive (i.e.
more adjective than the standard). The adjective is in the predicative
inflection.

\begin{sentence}
\shortex{3}
{\textbf{nna-j} &\textbf{wa-rna} &\textbf{yam-u} }
{3\textsc{s}-\textsc{nom} &1\textsc{s}-\textsc{rev} &tall-\textsc{an} }
{\textit{He is taller than me.}}
\end{sentence}

For comparatives in relative clauses, the adjective is fronted and is followed
by the arguments.

\begin{sentence}
\shortex{5}
{\textbf{k-a} &\textbf{nnalji-\o} &\textbf{junn-u} &\textbf{nna-\o} &\textbf{wiinguurki-rna} }
{win.\textsc{aux}-\textsc{pst} &dingo-\textsc{erg} &fast-\textsc{an} &3\textsc{s}-\textsc{erg} &boy-\textsc{rev} }
{\textit{}}
\shortex{3}
{\textbf{yuki-j} &\textbf{ka} &\textbf{giirr-u-m} }
{race-\textsc{nom} &2.\textsc{val}.1 &win-\textsc{an}-3\textsc{rd} }
{\textit{The dingo, who is faster than the boy, won the race.}}
\end{sentence}

\subsection{Conditional}\index{Conditional Phrases}

There are two types of conditionals: implicative and predictive. The protasis
(condition) and apodosis (outcome) are modified in different ways.

\begin{description}
\item[implicative] The conditional is a universal truth. Whenever the condition
  is true, the outcome is also true.
\item[predictive] The conditional is a prediction. If the condition occurs, the
  outcome will occur.
\end{description}

To form both conditionals, the condition verb phrase appears first, followed
immediately by the outcome verb phrase. There is no morpheme with equivalent
meaning to ``if''. However, the outcome is always placed in the subjunctive
mood and the present tense.

In an implicative conditional, the condition is given the gnomic mood. The
statement must therefore follow the usual rules of the gnomic, in that it must
state an undisputable truth. The condition is always in the present tense.

\begin{sentence}
\shortex{5}
{\textbf{k-i-nga} &\textbf{kunii-\o} &\textbf{mu-rn} &\textbf{naa} &\textbf{yarr-uu-n} }
{fall.\textsc{aux}-\textsc{prs}-\textsc{gno} &2\textsc{dual}.\textsc{ch}-\textsc{erg} &water-\textsc{loc} &1.\textsc{val}.2 &fall-\textsc{ch}-2\textsc{nd} }
{\textit{}}
\shortex{4}
{\textbf{j-i-tirlu} &\textbf{kunii-j} &\textbf{ka} &\textbf{mulj-awuu-n} }
{wet.\textsc{aux}-\textsc{prs}-\textsc{sbjv} &2\textsc{dual}-\textsc{ch}-\textsc{nom} &2.\textsc{val}.1 &wet-\textsc{ch}-2\textsc{nd} }
{\textit{If you two fall in the water, you will both get wet.}}
\end{sentence}

In a predictive conditional, the condition is usually not given a mood.
However, if the phrase is counterfactual, in that the condition is not seen as
likely, the condition occurs in the dubitative mood. Usually, the condition
will be in the future tense.

\begin{sentence}
\shortex{5}
{\textbf{nn-uuki} &\textbf{palwuuwa-j} &\textbf{ka} &\textbf{girnn-aa-mi} &\textbf{k-i}}
{break.\textsc{aux}-\textsc{fut} &branch-\textsc{nom} &2.\textsc{val}.1 &break-\textsc{inan}-3\textsc{rd} &strike.\textsc{aux}-\textsc{pres}  }
{\textit{}}

\shortex{4}
{\textbf{yannu-\o} & \textbf{nna-j} &\textbf{ji} &\textbf{wurr-a-rn} }
{\textsc{dem}.\textsc{sg}.\textsc{inan}-\textsc{erg} & 3\textsc{s}-\textsc{an}-\textsc{nom} &0.\textsc{val}.2 &strike-\textsc{inan}-3\textsc{rd} }
{\textit{If that branch breaks, it will strike him.}}
\end{sentence}

\begin{sentence}
\shortex{5}
{\textbf{k-aa-tila} &\textbf{nna-\o} &\textbf{maaju-j} &\textbf{yirn} &\textbf{parr-u-m}}
{hunt.\textsc{aux}-\textsc{fut}-\textsc{dub} &3\textsc{s}.\textsc{an}-\textsc{erg} &kangaroo-\textsc{nom} &completedly &hunt-\textsc{an}-3\textsc{rd}  }
{\textit{}}

\shortex{5}
{\textbf{ngarr-tiru} &\textbf{nurtwa-nurtwa-rn} &\textbf{yuni} &\textbf{waya-j} &\textbf{ngurr-a-m} }
{be.\textsc{aux}-\textsc{subj} & food-\textsc{pl}-\textsc{loc} &lots &1\textsc{pl}-\textsc{nom} &be-\textsc{inan}-3\textsc{rd} }
{\textit{If he were to successfully hunt the kangaroo [unlikely], we would have lots of food.}}
\end{sentence}

\subsection{Negative}\index{Negative Phrases}\label{negative}

There are two types of negation: clausal, where the entire clause is negated,
and constituent, where one noun is negated.

The formation of the clausal negative requires the negative particle that
corresponds to the class of the clause's verb. In a standard negative clause,
the particle follows the verb's auxiliary. However, in imperative clauses it
precedes the auxiliary. Qualifiers such as ``never'' are used following the
sentence, as stand-alone utterances.

\begin{sentence}
\shortex{5}
{\textbf{k-a} &\textbf{tu} &\textbf{nna-\o} &\textbf{naarla} &\textbf{wiirr-u-m} }
{go.\textsc{aux}-\textsc{pst} &\textsc{neg} &3\textsc{s}.\textsc{an}-\textsc{erg} &there &go-\textsc{an}-3\textsc{rd} }
{\textit{He didn't go there.}}

\shortex{6}
{\textbf{ti} &\textbf{j-i-yuu} &\textbf{ku-j} &\textbf{waa} &\textbf{yanj-awu-n.} &\textbf{wulnni} }
{\textsc{neg} &steal.\textsc{aux}-\textsc{strimp} &2\textsc{s}-\textsc{nom} &3.\textsc{val}.1 &steal-\textsc{an}-2\textsc{nd}. &never }
{\textit{You must never steal.}}
\end{sentence}

The consituent negative is applicable to clauses using the verb ``to have''. It
is formed using the special argument \textit{tunna} in the comitative slot of
the verb.

\begin{sentence}
\shortex{5}
{\textbf{rr-i} &\textbf{gunnari-\o} &\textbf{guwa-guwa-j} &\textbf{tunna} &\textbf{gurr-a-m} }
{have.\textsc{aux}-\textsc{prs} &tree-\textsc{erg} &leaf-\textsc{pl}-\textsc{nom} &none &have-\textsc{inan}-3\textsc{rd} }
{\textit{The tree doesn't have any leaves.}}
\end{sentence}

\subsection{Reflexive/Reciprocal}\index{Reflexive/Reciprocal Phrases}

In reflexive clauses, the personal pronoun of the subject simply occupies the
object position in the usual case. However, the valence of the verb must be
decreased by one.

\begin{sentence}
\shortex{5}
{\textbf{k-i} &\textbf{Paya-\o} &\textbf{nna-j} &\textbf{ka} &\textbf{tiirr-u-m} }
{carefor.\textsc{aux}-\textsc{prs} &\textsc{p}aya-\textsc{erg} &3\textsc{s}.\textsc{an}-\textsc{nom} &2.\textsc{val}.1 &carefor-\textsc{an}-3\textsc{rd} }
{\textit{Paya cares for himself.}}
\end{sentence}

If the clause is reciprocal, which applies only to plural subjects, the personal
pronoun is still used except it takes the comitative case. The valence is also
still decreased by one.

\begin{sentence}
\shortex{5}
{\textbf{k-arlu} &\textbf{kuu-j} &\textbf{kuu-yi} &\textbf{ka} &\textbf{pirr-u-n} }
{see.\textsc{aux}-\textsc{rem} &2\textsc{pl}-\textsc{nom} &2\textsc{pl}-\textsc{com} &2.\textsc{val}.1 &see-\textsc{an}-2\textsc{nd} }
{\textit{You [plural] used to see each other.}}
\end{sentence}

\section{Gerunds}\index{Gerund}

The gerund of a verb serves two purposes. It can act in a way similar to the
English gerund, where the verb is used as a noun, or in a way similar to an
infinitive, meaning roughly ``in order to''.

The gerund is formed through nominalising the verb. The last vowel of the verb
is simply appended as a suffix.

When used in the nominal form, the gerund takes the appropriate noun case.

\begin{sentence}
\shortex{4}
{\textbf{k-arlu} &\textbf{wa-j} &\textbf{junnu} &\textbf{yuurr-u-\o} }
{like.\textsc{aux}-\textsc{rem} &1\textsc{s}-\textsc{nom} &swim.\textsc{ger} &like-\textsc{an}-1\textsc{st} }
{\textit{I used to like swimming.}}
\end{sentence}

In the infinitive form, the gerund is placed before the verb's auxiliary.

\begin{sentence}
\shortex{5}
{\textbf{parra} &\textbf{k-a} &\textbf{nni-j} &\textbf{naarla} &\textbf{wiirr-u-m} }
{hunt.\textsc{ger} &go.\textsc{aux}-\textsc{pst} &3\textsc{s}.\textsc{an}-\textsc{nom} &there &go-\textsc{an}-3\textsc{rd} }
{\textit{He went there to hunt.}}
\end{sentence}

\section{Causatives}\index{Causatives}

There are two forms of the causative. The first occurs when a single noun is
responsible for causing a verb phrase to occur. In this case, the comitative
causative is used. However, if an entire verb phrase is responsible, the
subjunctive purposive is used.

\subsection{Comitative Causative}

In the comitative causative, an extra argument is added to the verb phrase
without modifying the valence. The argument is the causer, and takes the former
subject's form (be it nominative or ergative). The causee, or the argument which
was formerly the subject, then takes the comitative case instead. The verb
remains in agreement with the former subject.

\begin{sentence}
\shortex{3}
{\textbf{j-a} &\textbf{turrayi-j} &\textbf{mu nnij-a-m} }
{capsize.\textsc{aux}-\textsc{pst} &canoe-\textsc{nom} &capsize-\textsc{inan}-3\textsc{rd}}
{\textit{The canoe capsized.}}

\shortex{4}
{\textbf{j-a} &\textbf{turrayi-yi} &\textbf{nna-j} &\textbf{mu nnij-a-m} }
{capsize.\textsc{aux}-\textsc{pst} &canoe-\textsc{com} &3\textsc{s}.\textsc{an}-\textsc{nom} &capsize-\textsc{inan}-3\textsc{rd} }
{\textit{He caused the canoe to capsize.}}

\shortex{5}
{\textbf{k-a} &\textbf{wa-\o} &\textbf{wuta-j} &\textbf{walu} &\textbf{gukarr-u-\o} }
{drop.\textsc{aux}-\textsc{pst} &1\textsc{s}-\textsc{erg} &axe-\textsc{nom} &my &drop-\textsc{an}-1\textsc{st} }
{\textit{I dropped my axe.}}
\shortex{6}
{\textbf{k-a} &\textbf{wa-yi} &\textbf{wuta-j} &\textbf{walu} &\textbf{gaju-\o} &\textbf{gukarr-u-\o} }
{drop.\textsc{aux}-\textsc{pst} &1\textsc{s}-\textsc{com} &axe-\textsc{nom} &my &wind-\textsc{erg} &drop-\textsc{an}-1\textsc{st} }
{\textit{The wind caused me to drop my axe.}}
\end{sentence}

\subsection{Subjunctive Purposive}\label{subjunctivepurposive}

The subjunctive purposive is formed through the use of the verb \textit{nnurr}
``to effect''. The auxiliary, \textit{nnarr} takes the present tense, and begins
the sentence. The verb itself is not required, but it still takes two verb
phrases as arguments. The verb phrase causing the other assumes its usual tense
and mood, but the caused action becomes present and subjunctive.

\begin{sentence}
\shortex{5}
{\textbf{nnarr-i} &\textbf{k-a} &\textbf{nna-\o} &\textbf{naarla} &\textbf{wiirr-u-m} }
{effect.\textsc{aux}-\textsc{prs} &go.\textsc{aux}-\textsc{pst} &3\textsc{s}.\textsc{an}-\textsc{erg} &there &go-\textsc{an}-3\textsc{rd} }
{\textit{}}
\shortex{4}
{\textbf{j-i-tirlu} &\textbf{wa-j} &\textbf{nna-l} &\textbf{nnurr-u-\o} }
{follow.\textsc{aux}-\textsc{prs}-\textsc{sbjv} &1\textsc{s}-\textsc{nom} &3\textsc{s}-\textsc{an}-\textsc{acc} &follow-\textsc{an}-1\textsc{st} }
{\textit{He went there, so I followed him.}}
\end{sentence}

\section{Subjunctive}\index{Subjunctive}

\subsection{Desires}

To express desires, a ``wanting'' verb is used, such as ``to dream'', along with
a verb phrase in the subjunctive expressing the desired action. The action can
be in any tense.

\begin{sentence}
\shortex{6}
{\textbf{nn-i} &\textbf{wa-j} &\textbf{j-a-tirlu} &\textbf{ti} &\textbf{nna-j} &\textbf{ngarj-awu-m}}
{wish.\textsc{aux}-\textsc{prs} &1\textsc{s}-\textsc{nom} &hurt.\textsc{aux}-\textsc{pst}-\textsc{sbjv} &\textsc{neg} &3\textsc{s}.\textsc{an}-\textsc{nom} &hurt-\textsc{an}-3\textsc{rd}}
{\textit{}}
\shortex{1}
{\textbf{mann-u-\o} }
{wish-\textsc{an}-1\textsc{st} }
{\textit{I wish that he hadn't hurt himself}}
\end{sentence}

\subsection{Speculation}\index{Speculation}

If the speaker is speaking hypothetically about a situation, the subjunctive can
be used. In this case, the verb ``to be'' would be used with a predicate
adjective rather than the verbless construction.

\begin{sentence}
\shortex{6}
{\textbf{ngarr-aa-tilu} &\textbf{parra-\o} &\textbf{kurlu-j} &\textbf{tuwilwa-wa} &\textbf{ka} &\textbf{ngurr-a-m} }
{be.\textsc{aux}-\textsc{fut}-\textsc{sbjv} &hunt-\textsc{erg} & thing-\textsc{NOM} & dangerous-\textsc{in} &2.\textsc{val}.1 &be-\textsc{inan}-3\textsc{rd} }
{\textit{The [prospective] hunt would be very dangerous.}}
\end{sentence}
