\chapter{Syntax}

\section{Alignment}\index{Morphosyntactic Alignment}

The alignment of Ngujari depends on whether the noun in question is an animate
pronoun or not. For clauses with exclusively animate pronouns, the alignment is
nominative-accusative, but otherwise it is ergative-nominative (i.e. the
transitive patient and intransitive object are marked nominative and the
transitive agent is marked ergative). This system applies only to intransitive
and transitive verbs. For higher valencies, formed through \ref{valencemod}, the
extra arguments are assigned cases semantically.

\section{Verb Phrases}\index{Verb Phrases}

\begin{definition}[Verb Phrase]
~\\
\textsc{vp = aux [neg] np(s) [adv(s)] [val] v}
\end{definition}

Verb phrases can be as simple as a single avalent verb, such as in ``it's raining'', or as complex as a tetravalent causative.

In the prototypical verb clause, the following rules govern word order:

\begin{enumerate}
\item The verb's auxiliary appears at the beginning.
\item The verb itself appears at the end.
\item Valence modifiers appear immediately before the verb.
\end{enumerate}

The following examples illustrate basic verb phrases:

\begin{sentence}
\shortex{5}
{\bfseries Kuurl & \bfseries wa-wa & \bfseries kurru-rru & \bfseries ji &
  \bfseries wurr-u-\o.}
{\textsc{aux} & 1s-\textsc{nom} & 2s-\textsc{ACC} & \textsc{0.val.2} & electrically.storm-\textsc{an}-\textsc{1st}}
{\textit{I strike you}}

\shortex{4}
{\bfseries Wann-uma & \bfseries maaju~maaju-wa & \bfseries ka & \bfseries jinn-u-ni.}
{\textsc{aux-pst} & kangaroo-\textsc{pl-nom} & \textsc{2.val.1} & eat-\textsc{an-3rd}}
{\textit{The kangaroos ate/were eating.}}
\end{sentence}

Noun phrases tend to appear in order of importance to the statement as judged by
the speaker.

\section{Noun Phrases}\index{Noun Phrases}

\begin{definition}[Verb Phrase]
~\\
\textsc{np = [adj(s)-attr] n [rel(s)]}
\end{definition}

A noun phrase consists of one noun, declined by case, and any number of
adjectives and relative clauses. The noun tends to be placed first, followed by
adjectives, although this can be inverted or even mixed according to pragmatic
considerations. However, relative clauses always succeed the noun and
adjectives.

\begin{sentence}

\shortex{3}
{\textbf{birru-\o} &\textbf{birruku} &\textbf{miinna} }
{sea-\textsc{ERG} &blue &big}
{\textit{vast blue sea}}

\shortex{3}
{\textbf{kanaama} &\textbf{yirlirna-wa} &\textbf{gu} }
{woven &basket-\textsc{NOM} &small}
{\textit{small woven basket}}

\end{sentence}

\section{Relative Clauses}\index{Relative Clauses}\label{relativeclauses}

\begin{definition}[Relative Clause]
~\\
\textsc{vp = aux [neg] np(s) [adv(s)] [val] v}\\
$\Rightarrow$ \textsc{rc = aux [neg] v [val] [adv(s)] np(s)}
\end{definition}

Relative clauses are \textit{adjoined} to the noun phrase. The clause undergoes
a transformation from the standard verb phrase by moving the verb to the
position immediately following the auxiliary. The valency modifier is free to be
placed anywhere among the remaining noun phrases and adverbs, but typically
follows the verb.

If the head noun is a patient of the relative clause, the verb of the relative
clause has its valence reduced by one.

\begin{sentence}
\shortex{5}
{\textbf{gungi-\o} &\textbf{kuurl-a} &\textbf{pirr-u-\o} &\textbf{ka} &\textbf{wawa}}
{man-\textsc{erg} &\textsc{aux}-\textsc{pst} &see-\textsc{an}-1\textsc{st} &2.\textsc{val}.1 &1s-\textsc{nom}}
{\textit{the man that I saw}}
\end{sentence}

If the head noun is the agent, a pronoun is used inside the relative clause to
refer back to it.

\begin{sentence}
\shortex{6}
{\textbf{ngiy-a} &\textbf{Wuurna-\o} &\textbf{wann-aju-ti} &\textbf{yann-u-ni} &\textbf{nna-wa} &\textbf{jurlu-rru}}
{\textsc{aux}-\textsc{pst} &Wuurna-\textsc{erg} &\textsc{aux}-\textsc{fut}-\textsc{dub} &catch-\textsc{an}-3\textsc{rd} &3s-\textsc{nom} &turtle-\textsc{acc}}
{}\\
\shortex{3}
{\textbf{wa-wa} &\textbf{ka} &\textbf{naj-u-ni} }
{1s-\textsc{nom} &3.\textsc{val}.2 &say-\textsc{an}-3\textsc{rd} }
{\textit{Wuurna, who might catch a turtle, spoke to me.}}

\end{sentence}

\subsection{Adverbial Phrases}\index{Adverbial Phrases}
\label{advsyntax}

Temporal adverbs, which specify the time an action takes place, tend to appear
following the noun.

\begin{sentence}
\shortex{6}
{\textbf{nuuj-a} &\textbf{jana-\o} &\textbf{jari-ru} &\textbf{wiirr-uu-\o} &\textbf{yuurli-nga} &\textbf{ma} }
{go.\textsc{aux}-\textsc{pst} &1s.\textsc{ch}-\textsc{erg} &beach-\textsc{loc} &go-\textsc{ch}-1\textsc{st} &day-\textsc{rev} &one }
{\textit{Yesterday, I [a child] went to the beach.}}
\end{sentence}

Manner adverbs, which specify the manner in which the action was conducted,
usually appear directly before the noun.

aux(topickup)-weakimp 1pl-ERG clothes(pl)-NOM quickly pickup-an-3rd.

We should pick up the clothes quickly.

However, both can occupy different positions inside the verb phrase if the
speaker desires it.

\section{Predicates}\index{Predicates}

There are three cases for predicates: adjectival, nominal, and locational.

In an adjectival predicative phrase a verb is not normally required. The noun is
assigned the same tense as it would be were it the argument to an intransitive
verb, while the adjective assumes its predicative inflection.

sky-NOM blue-PRED.

the sky is blue.

Sometimes, the comitative case is used along with the verb ``to be'' in an
adjectival phrase, usually when describing a changeable state.

berry-NOM freshness-COM.

the berry is with freshness/is fresh.

In a nominal predicative phrase, the verb ``to be'' is used. The predicate noun
is declined as verb's object.

aux(tobe) 1s-ERG teacher-NOM tobe-an-1st.

I am a teacher.

In a locational predicative phrase, the verb ``to be'' is still used, but the
predicate location is declined in the locative case.

aux(tobe) village-ERG somewhere-LOC tobe-inan-3rd.

The village is somewhere.

\section{Possession}\index{Posession}

\subsection{Alienable}

To indicate alienable possession (possession that is not permanent or subject to
change), the locative case is used in conjunction with the verb ``to be''. The
possessed noun appears in the locative case as the subject of the transitive
form of ``to be'', with the possessor appearing as the object in the usual case.

aux deadfish-pl(dual)-locative woman-NOM is-inanimate-3rd.

The woman has two dead fish.

\subsection{Inalienable}

Inalienable possession (possession that is unequivocal) is indicated simply
through the use of the verb ``to have''.

aux-fut-gnomic mob-nom homeland-acc valence3->2 have-an-1st.

Our mob will always have a homeland.

\subsection{Pronominal}

A noun phrase can be indicated as possessed through the use of a possessive
pronoun as an adjective.

aux-past 3pl-an-ERG face-NOM beautiful his admire-an-3rd

they admired his beautiful face

In Ngujari culture, an object can be owned by a mob as a whole. Only inanimate
objects may be possessed by a mob (with the exception of areas of land).
Possession is indicated by the particle \textit{tuu}, which appears before the noun. To
specify the possessing mob, the mob's name is placed immediately after the
particle. The regular name is used by members of the possessing mob, but the
honorific name is used for possessions of others. For example, the particle for
something owned by the Wujanga mob would be \textit{tuu-Wujanga} for a member or
\textit{tuu-Wujarra} for an outsider.

aux-strongimp 1pl-ERG tuu-Wujanga precious land-NOM spirit-INST valence2->3
protect-an-1st

we must protect our (the Wujanga mob's) precious land with vigour

\section{Verbal Constructions}\index{Verbal Constructions}

\subsection{Interrogative}\index{Interrogative}

\subsubsection{Polar Questions}

Polar questions are syntactically the same as a factual statement, except they
are expressed with a rising tone at the beginning of the question.

rise aux-future bird-pl-ERG mountain-pl-LOC fly-an-3rd.

Will the birds fly to the mountains?

\subsubsection{Non-Polar Questions}

One way of forming a non-polar question is using an interrogative pronoun as a
verb's argument, with no syntactic change taking place.

aux1 path-NOM aux2 2leadto-in-3rd 3s-inan-ERG village-NOM kiru 1tobe-in-3rd.

the path that leads to the village is where

Where is the path to the village?

To question a certain word in a statement, the particle *yuu* can be placed
before the word.

aux-future yuu-3dual-ERG food-NOM fire-LOC bring-an-3rd?

Will *those two* bring the food to the fire?

aux 3s-ERG yuu-fresh kangaroomeat-NOM eat-an-3rd?

Is he eating *fresh* kangaroo meat?

\subsection{Comparative}\index{Comparative Phrases}

Ngujari contains locational-type comparatives. This means that the *standard*
noun, or the noun to be judged against, is marked in the revertive case.
Comparatives do not use a verb, and are always positive (i.e. more adjective
than the standard). The adjective is in the predicative inflection.

3an-s-NOM 1s-REV tall-PRED

He is taller than me.

For comparatives in relative clauses, the adjective is fronted and is followed
by the arguments.

aux-past dingo-ERG [fast-PRED 3an-s-ERG boy-REV] race-NOM valence1->2
race-an-3rd

The dingo, who is faster than the boy, won the race.

\subsection{Conditional}\index{Conditional Phrases}

There are two types of conditionals: implicative and predictive. The protasis
(condition) and apodosis (outcome) are modified in different ways.

\begin{description}
\item[implicative] The conditional is a universal truth. Whenever the condition
  is true, the outcome is also true.
\item[predictive] The conditional is a prediction. If the condition occurs, the
  outcome will occur.
\end{description}

To form both conditionals, the condition verb phrase appears first, followed
immediately by the outcome verb phrase. There is no morpheme with equivalent
meaning to ``if''. However, the outcome is always placed in the subjunctive
mood and the present tense.

In an implicative conditional, the condition is given the gnomic mood. The
statement must therefore follow the usual rules of the gnomic, in that it must
state an undisputable truth. The condition is always in the present tense.

aux-gnomic 2dual-ERG water-LOC valence1->2 fall-an-2nd, aux-subj 2dual
valence2->1 towet-an-2nd.

If you two fall in the water, you will both get wet.

In a predictive conditional, the condition is usually not given a mood.
However, if the phrase is counterfactual, in that the condition is not seen as
likely, the conditon occurs in the dubitative mood. Usually, the condition
will be in the future tense.

aux-fut branch-NOM valence2->1 break-in-3rd, aux yannu(sing inan demon)-ERG
3s-an-NOM valence0->2 toelectricallystorm-in-3rd

If that branch breaks, it will strike him.

aux-dub-fut 3s-an-ERG kangaroo-NOM successfullyhunt-an-3rd, aux-SUBJ food-pl-LOC
lots 1pl-NOM tobe-in-3rd.

If he were to successfully hunt the kangaroo (unlikely), we would have plenty of
food.

\subsection{Negative}\index{Negative Phrases}

There are two types of negation: clausal, where the entire clause is negated,
and constituent, where one noun is negated.

The formation of the clausal negative requires the negative particle that
corresponds to the class of the clause's verb. In a standard negative clause,
the particle follows the verb's auxiliary. However, in imperative clauses it
precedes the auxiliary. Qualifiers such as ``never'' are used following the
sentence, as stand-alone utterances.

aux-past neg 3-an-s there togo-an-3rd.

He didn't go there.

neg aux-strongimperative 2s valence3->1 steal-an-2nd. Never.

you must never steal.

The consituent negative is applicable to clauses using the verb ``to have''. It is
formed using the special argument \textit{tunna} in the comitative slot of the verb.

aux tree-ERG leaf-pl-NOM tunna have-inan-3rd.

the tree doesn't have any leaves.

\subsection{Reflexive/Reciprocal}\index{Reflexive/Reciprocal Phrases}

In reflexive clauses, the personal pronoun of the subject simply occupies the
object position in the usual case. However, the valence of the verb must be
decreased by one.

aux Paya-ERG 3s-an-NOM valence2->1 carefor-an-3rd.

Paya cares for himself.

If the clause is reciprocal, which applies only to plural subjects, the personal
pronoun is still used except it takes the comitative case. The valence is also
still decreased by one.

aux-remote 2pl-NOM 2pl-COM valence2->1 see-an-2nd.

You(pl) used to see each other.

\section{Causatives}\index{Causatives}

There are two forms of the causative. The first occurs when a single noun is
responsible for causing a verb phrase to occur. In this case, the comitative
causative is used. However, if an entire verb phrase is responsible, the
subjunctive purposive is used.

\subsection{Comitative Causative}

In the comitative causative, an extra argument is added to the verb phrase
without modifying the valence. The argument is the causer, and takes the former
subject's form (be it nominative or ergative). The causee, or the argument which
was formerly the subject, then takes the comitative case instead. The verb
remains in agreement with the former subject.

aux-past canoe-NOM capsize-in-3rd.

The canoe capsized.

aux-past canoe-COM 3s-an-NOM capsize-in-3rd.
He caused the canoe to capsize.

aux-past 1s-ERG axe-NOM my drop-an-1st.

I dropped my axe.

aux past 1s-COM axe-NOM wind-ERG drop-an-1st.

The wind caused me to drop my axe.

\subsection{Subjunctive Purposive}

The subjunctive purposive is formed through the use of the verb ``to effect''.
The verb takes two verb phrases as arguments. The verb phrase causing the other
assumes its usual tense and mood, but the caused action becomes present and
subjunctive.

aux(to effect) aux(to go)-PAST 3s-an-ERG there-LOC togo-an-3rd aux(follow)-SUBJ
1s-NOM 3s-an-ACC follow-an-1st.

He went there so I followed him.

\section{Subjunctive}\index{Subjunctive}

\subsection{Desires}

To express desires, a ``wanting'' verb is used, such as ``to dream'', along with
a verb phrase in the subjunctive expressing the desired action. The action can
be in any tense.

aux(towish) 1s-NOM aux(becomehurt) neg 3s-an-NOM becomehurt-an-3rd wish-an-1st.

I wish that he hadn't hurt himself.

\subsection{Speculation}\index{Speculation}

If the speaker is speaking hypothetically about a situation, the subjunctive can
be used. In this case, the verb ``to be'' would be used with a predicate
adjective rather than the verbless construction.

aux(tobe)-SUBJ-FUT hunt-ERG dangerous-PRED valence2->1 tobe-inan-3rd.

(speaking about a prospective hunt) The hunt would be very dangerous.
