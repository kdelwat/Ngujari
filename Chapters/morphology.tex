\chapter{Morphology}

\section{Nouns}\index{Nominal Morphology}

\subsection{Gender}\index{Gender}

Ngujari has four genders: child, adult, elder (grouped together as animate), and
inanimate. Gender is assigned semantically and changes the morphosyntactic
alignment of the sentence as well as posessives.

The animate gender is given to people, animals, and Dreamtime figures. For
example, \textit{Yawirra}, the concept of the Land, is considered animate. The
inanimate gender applies to all other nouns.

Within the animate there are three genders, each representing a different stage
in life. This distinction is important in areas such as pronouns, but not in
others, like verbal inflection. An animate noun is assigned to a stage based on
their social position. Those who are yet to undergo the adulthood ceremony
(those under roughly 14 in the case of females and 16 in the case of males) are
assigned the child gender, while those who have become elders receive the elder
gender. All other ages are grouped into the adult gender.

\subsection{Cases}\index{Cases}

Ngujari has eight nominal cases, with three indicating the morphosyntactic
alignment and five others. Cases are indicated by single-syllable suffixes, as
indicated in the following table.

\begin{table}[h]
\centering
\begin{tabular}{lcc}
\textbf{case} & \textbf{abbreviation} & \textbf{suffix}\\
ergative & \textsc{erg} & -\\
nominative & \textsc{nom} & -wa\\
accusative & \textsc{abs} & -rru\\
instrumental & \textsc{ins} & -ma\\
comitative & \textsc{com} & -yii\\
orientative & \textsc{ori} & -rni\\
revertive & \textsc{rev} & -nga\\
locative & \textsc{loc} & -ru\\
\end{tabular}
\caption{Case Suffixes}
\end{table}

For more details on the three alignment cases, see \ref{sec:alignment} (pg.
\pageref{sec:alignment}). The remaining five cases operate as follows:

\begin{description}
\item[instrumental] The instrumental case is used when discussing a *means*,
  roughly equivalent to the English ``by means of''. For example, when speaking
  of killing a fish using a spear, a Ngujari speaker will place ``spear'' in the
  \textsc{INS} case.
\item[comitative] The comitative case is equivalent to ``in the presence of'',
  or ``with'', and specifies that the noun was present at the moment spoken of.
\item[orientative] The orientative case is used to specify that something is
  facing towards the noun. It is often used with the meaning of ``heading
  towards''.

  aux 2s-ERG camp-ORI togo-an-2nd.

  You are heading towards the camp.

\item[revertive] The revertive case is used to specify that something is
    oriented away from the noun. It can be used with the meaning of ``heading
    away from''.

  aux 3pl-an-NOM 3s-an-REV togo-an-3rd.

  They are heading away from her.

  It can also be used in asserting falsehood.

  aux-remote 3s-an-ERG knowledge-NOM valence1->2 tolook-an-3rd.

  He used to look away from knowledge / he used to be incorrect.

\item[locative] The locative case is used to specify a location, and can take
  the place of a preposition such as ``in'' or ``at''. This means that ``she is
  at the house'' is equivalent to ``she is [house] (\textsc{LOC})''. The
  locative suffix *-ru* becomes a long u if placed after a word ending in a
  short u.

\end{description}

An example of the use of these cases is found in the following table, which
shows the declensions of the noun \textit{naju}, or ``rock''.

\begin{table}[h]
\centering
\begin{tabular}{lll}
\textbf{case} & \textbf{word} & \textbf{meaning}\\
ergative & naju & -\\
nominative & najuwa & -\\
accusative & najurru & -\\
instrumental & najuma & ``using the rock''\\
comitative & najuyii & ``in the presence of the rock''\\
orientative & najurni & ``oriented towards the rock''\\
revertive & najunga & ``orientated away from the rock''\\
locative & najuu & ``at the rock''\\
\end{tabular}
\caption{Examples of Nominal Case Declensions}
\end{table}

\section{Plurality}\index{Plurality}

Plurals are formed through reduplication, with the declined noun repeated twice.
For example, *najurru* ("rock", in the absolutive case), would be pluralised as
*najurru-najurru*.

There are two forms of plural, which differentiate dual and non-dual plurality.
The default case is non-dual, but the clitic *ka* following the reduplicated
noun indicates the dual form.

\section{Verbs}\index{Verbal Morphology}

Verbs in Ngujari are found in three classes, each with a specified stem ending
and auxiliary form. Verb roots lack a final consonant, meaning they must be
conjugated in order to appear in speech. Class does not have any semantic
impact; it changes only the morphology of the verb.

The three classes are:

\begin{table}[h]
\centering
\begin{tabular}{lccc}
\textbf{class} & \textbf{ending} & \textbf{auxiliary} & \textbf{negative particle}\\
\textbf{first} & -rr & kuurl & tu\\
\textbf{second} & -j & ngiy & ti\\
\textbf{third} & -nn & wann & wuu\\
\end{tabular}
\caption{Verb Classes}
\end{table}

To conjugate a verb, both it and its auxiliary must be declined. The verb itself
is conjugated in agreement, with the gender and person of the subject indicated
as affixes. The auxiliary is declined to indicate tense and mood.

\subsection{Tense and Mood}\index{Tense and Mood}

There are four tenses: remote past, past, present, and future. There is no
distinction drawn between the perfective and imperfective aspects, meaning
contextual clues are vital for understanding.

Present is considered the default tense, and is accordingly unmarked for first
and second class verbs (but not third). It usually indicates those events which
are happening in the moment of utterance, but it can also be used as a
rudimentary form of a near-past tense, applying to actions that were completed
the same day as the utterance.

Past and remote past are marked for all verb classes and indicate an event that
was completed in the past. Choice between the two can be somewhat arbitrary, but
in general remote past is used when recounting handed-down stories or the events
of ancestral times, whereas basic past refers to events in the time period of
the speaker. If the event has not yet finished, the present tense is used.

Future is again marked for all classes. All events which are yet to take place
are assigned the future tense.

There are five moods that a verb can optionally be conjugated for:

\begin{itemize}
\item subjunctive
\item weak imperative
\item strong imperative
\item gnomic
\item dubitative
\end{itemize}

\begin{description}

\item[subjunctive] The subjunctive mood is an irrealis mood which broadly
  signifies abstractness and is used in a number of ways:

  \begin{enumerate}
      \item Speculation
      \item Conditional
      \item Desires
      \item Purposive
  \end{enumerate}

\item[imperative] The imperative mood is used for suggestions and commands. The
  weak form raises an idea without indicated an order, similar to the English
  ``let's go'', whereas the strong form signifies a command, such as ``Leave!''.

\item[gnomic] The gnomic mood states unequivocal facts or ideas. The statement
  must be truly uncontentious to fit into the gnomic mood, such as ``fire is
  real''.

\item[dubitative] The dubitative mood indicates situational possibility, in that
  the speaker acknowledges the possibility of an action but is unsure as to
  whether it occurs, as in English ``might''.

\end{description}

\subsection{Verbal Conjugation Tables}

\begin{table}[h]
\centering
\begin{tabular}{lcccc}
\textbf{class} & \textbf{child} & \textbf{adult} & \textbf{elder} & \textbf{inanimate}\\
first & uu & u & iiwa & a\\
second & awuu & awu & iwu & a\\
third & arruu & u & iwu & aa\\
\end{tabular}
\caption{Gender of Subject}
\end{table}

\begin{table}[h]
\centering
\begin{tabular}{lccc}
\textbf{class} & \textbf{1st} & \textbf{2nd} & \textbf{3rd} \\
first, second & -           & ku          & nni    \\
third & -           & ku          & ni    \\
\end{tabular}
\caption{Person of Subject}
\end{table}

\subsection{Auxiliary Conjugation Tables}

\begin{table}[h]
\centering
\begin{tabular}{lcccc}
\textbf{class} & \textbf{remote past} & \textbf{past} & \textbf{present} & \textbf{future}\\
first & arlu & a & --- & aa \\
second & arlu & a & --- & aju\\
third & una & uma & uu & uuki\\
\end{tabular}
\caption{Tense}
\end{table}

\begin{table}[h]
\centering
\begin{tabular}{lccccc}
\textbf{class} & \textbf{subjunctive} & \textbf{weak imperative} & \textbf{strong imperative} & \textbf{gnomic} & \textbf{dubious}\\
first &  tiru & yii & ju & nga & tila\\
second & tirlu & yii & yuu & nga & ti\\
third &  tirlu & yii &  aru & nga & ti\\
\end{tabular}
\caption{Mood}
\end{table}

\subsection{Valence Modification}\index{Valence Modification}

The verbal system of Ngujari allows for many different valences through
derivations of base verbs. Each verb root has its own \textit{default valence},
between avalent (0 arguments) to quadrivalent (4 arguments). Furthermore, each
verb has a \textit{minimum valence} and \textit{maximum valence}, i.e. the
extent that valency can be modified while still modifying the verb's meaning,
rather than imparting additional information. The maximum valence is never above
4.

For example, the verb \textit{wurr} has a default valence of 0, in which case it
means ``it is electrically storming''. However, modifying its valence to 1
allows it to mean ``to be struck by lightning'', and a valence of 2 allows it to
mean ``to strike''. Therefore, it has a minimum valence of 0 and maximum valence
of 2.

Valence modification occurs through special particles, which are found in the
following table:

\begin{table}[h]
\centering
\begin{tabular}{lllllll}
                                  &   & \multicolumn{5}{c}{\textbf{target}} \\
                                  &   & 0     & 1     & 2   & 3     & 4     \\
\multirow{5}{*}{\textbf{default}} & 0 & ---   & wi    & ji  & murnu & yurnu \\
                                  & 1 & wi    & ---   & naa & naki  & mu    \\
                                  & 2 & waa   & ka    & --- & naa   & naki  \\
                                  & 3 & wangu & waa   & ka  & ---   & naa   \\
                                  & 4 & wirru & wangu & waa & ka    & ---
\end{tabular}
\caption{Valence Modification Particles}
\end{table}

The prime function of derived valences is to change the meaning of the verb. In
this case, the new meaning must be learned, as well as the noun cases it
accepts.

\section{Adjectives and Adverbs}\index{Adjectives and Adverbs}

Adjectives are inflected into two categories: attribute and predicate. The
attributive form is unmarked, and can be used directly in noun phrases to
describe the noun. The predicate form can only be used in predicative phrases,
and is declined according to the gender of the noun it applies to.

To decline a predicate adjective, the final vowel is dropped and the same
gender declension as followed by class one verbs is applied.

Adverbs are not declined, but are divided semantically into the classes manner
(hastily, carefully) and temporal (last week, yesterday). The class of an adverb
loosely determines its position in a phrase. See \ref{adverbposition} for more
information.
