\chapter{Derivation}

There are two types of derivation in Ngujari. The first, lexical derivation, is
not applicable \textit{generally}, and is instead used during word formation to
derive new vocabulary. While the patterns are regular, the application is not.
On the other hand, general derivation allows the modification of existing words,
including changing part of speech.

\section{Lexical Derivation}\index{Lexical Derivation}

\subsection{Verbs}

As verb roots are a closed class, derivation is the only way to form new verbs.
This method is known as \textbf{compounding}.

\subsubsection{Compounding}

There are two forms of compounding: verb-verb and adverb-verb. Both form a new
verb which is treated as a whole in syntactic structures.

In verb-verb compounding, the compound is not commutative, meaning that the
order of the verbs matters. Typically, the most relevant verb occurs last. The
two verbs are simply concatenated, except for the special case in which the
concatenation would form an illegal consonant cluster. If this occurs, the
repair strategy of inserting the dummy vowel \textit{a} is used.

\begin{quote}
\begin{multicols}{2}
to sit \textit{ngurr-}\\
to swim \textit{junn-}\\
$\Rightarrow$ to canoe \textit{ngurrjunn-}

to travel \textit{nuunn-}\\
to exchange \textit{murr-}\\
$\Rightarrow$ to trade (with another mob)~\textit{nuunnamurr-}
\end{multicols}
\end{quote}

Verbs formed through verb-verb compounding in most cases assume the transitivity
properties of the second, or primary, verb.

Adverb-verb compounding simply requires an adverb to appear before the verb in
all positions. For example, it would remain in front of the verb during
relativization (see~\ref{relativeclauses}) while regular adverbs would not.

\begin{quote}
\begin{multicols}{2}
to run \textit{yaj-}\\
quickly \textit{garrna}\\
$\Rightarrow$ to sprint \textit{garrna yaj-}

to drink \textit{ngann-}\\
impatiently \textit{karlpii}\\
$\Rightarrow$ to guzzle \textit{karlpii ngann-}
\end{multicols}
\end{quote}

In the case of adverb-verb compounding, the auxiliary of the verb may change so
that it does not match the verb class' standard form. If the adverb ends in a
lengthened vowel, the first vowel of the auxiliary becomes lengthened if it is
not already. For example, the auxiliary for \textit{karlpii ngann} (``to
guzzle'') changes from the standard \textit{wann} to \textit{waann}.

\subsection{Nouns}

There are many noun derivational operations.

\subsubsection{Compounding}

Nominal compounds are bidirectional, meaning that the order of constituent nouns
does not change the meaning of the compound. In practical use, both orders are
used, with preference depending on the phonetics of the noun. If the compounding
would create an illegal consonant cluster, the other order must be used.

\begin{quote}
\begin{multicols}{2}
mountain \textit{gaypa}\\
stream \textit{munna}\\
$\Rightarrow$ mountain stream \textit{gaypamunna}

the Land \textit{Yawirra}\\
ground \textit{nnalu}\\
$\Rightarrow$ sacred place \textit{Yawirrannalu}
\end{multicols}
\end{quote}

\subsubsection{Collection}

Partial reduplication can be used to derive the collection of a noun. To derive
the collection, the first syllable is isolated, its coda removed, and added to
the front of the noun.

\begin{quote}
\begin{multicols}{2}
coconut \textit{wurna}\\
$\Rightarrow$ coconut tree \textit{wuwurna}

bone \textit{parrna}\\
$\Rightarrow$ corpse \textit{paparrna}
\end{multicols}
\end{quote}

\subsubsection{Container}

The container of a noun can be derived through the affix \textit{rna}.

\begin{quote}
\begin{multicols}{2}
arrow \textit{yungi}\\
$\Rightarrow$ quiver \textit{yungirna}

fruit \textit{yirli}\\
$\Rightarrow$ basket \textit{yirlinga}
\end{multicols}
\end{quote}

\subsection{Adjectives}

\subsubsection{Cases}

Noun case suffixes can in some cases be used to derive adjectives based around
that noun. The most common forms of case derivations are orientative/revertive
and instrumental.

Orientative and revertive suffixs can be used to indicate the ``direction'' of
an adjective in relation to its noun. This distinction is commonly found when
talking about time.

\begin{quote}
\begin{multicols}{2}
age \textit{jul}\\
$\Rightarrow$ new \textit{jurni} (orientative)\\
$\Rightarrow$ old \textit{julnga} (revertive)

freshness \textit{gurlu}\\
$\Rightarrow$ fresh \textit{gurlurni}\\
$\Rightarrow$ stale \textit{gurlunga}
\end{multicols}
\end{quote}

\section{General Derivation}\index{General Derivation}

\subsection{Adjectives}

\subsubsection{Negation}

An adjective can be negated through a prefix. If the adjective begins with a
stop, the prefix is \textit{wuu}. Otherwise, it is \textit{tu}.

\begin{quote}
\begin{multicols}{2}
long (distance) \textit{yungi}\\
$\Rightarrow$ short (distance) \textit{tuyungi}

heavy (rain) \textit{ganu}\\
$\Rightarrow$ light (rain) \textit{wuuganu}
\end{multicols}
\end{quote}

\subsubsection{Amplification}

An adjective can be amplified in magnitude through reduplication. The final
syllable is duplicated, excluding its coda in the first instance. If the final
vowel is long following the derivation, it becomes shortened.

\begin{quote}
\begin{multicols}{3}
big \textit{yampu}\\
$\Rightarrow$ enormous \textit{yampupu}

small \textit{pangii}\\
$\Rightarrow$ tiny \textit{pangiigi}

thick \textit{yurlan}\\
$\Rightarrow$ fat \textit{yurlarlan}
\end{multicols}
\end{quote}

\subsubsection{Relativisation}

Many of Ngujari's adjectives are absolute rather than relative. For example,
\textit{yampu} (``big'') refers to something bigger than a human, rather than
something big for its class (as in ``the big elephant''). These absolute
adjectives can be converted to relative adjectives through the suffix
\textit{pu}.

\begin{quote}
\begin{multicols}{2}
small (absolute) \textit{pangii}\\
$\Rightarrow$ small (relative) \textit{pangiipu}

warm (absolute) \textit{mirra}\\
$\Rightarrow$ warm (relative) \textit{mirrapu}
\end{multicols}
\end{quote}
