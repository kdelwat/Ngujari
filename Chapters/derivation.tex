\chapter{Derivation}

\section{Verbs}\index{Verbal Derivation}

As verb roots are a closed class, derivation is the only way to form new verbs.
This method is known as \textbf{compounding}.

\subsection{Compounding}

There are two forms of compounding: verb-verb and adverb-verb. Both form a new
verb which is treated as a whole in syntactic structures.

In verb-verb compounding, the compound is not commutative, meaning that the
order of the verbs matters. Typically, the most relevant verb occurs last. The
two verbs are simply concatenated, except for the special case in which the
concatenation would form an illegal consonant cluster. If this occurs, the
repair strategy of inserting the dummy vowel \textit{a} is used.

\begin{quote}
\begin{multicols}{2}
to sit \textit{ngurr-}\\
to swim \textit{junn-}\\
$\Rightarrow$ to canoe \textit{ngurrjunn-}

to travel \textit{nuunn-}\\
to exchange \textit{murr-}\\
$\Rightarrow$ to trade (with another mob)~\textit{nuunnamurr-}
\end{multicols}
\end{quote}

Adverb-verb compounding simply requires an adverb to appear before the verb in
all positions. For example, it would remain in front of the verb during
relativization (see~\ref{relativeclauses}) while regular adverbs would not.

\begin{quote}
\begin{multicols}{2}
to run \textit{yaj-}\\
quickly \textit{garrna}\\
$\Rightarrow$ to sprint \textit{garrna yaj-}

to drink \textit{ngann-}\\
impatiently \textit{karlpii}\\
$\Rightarrow$ to guzzle \textit{karlpii ngann-}
\end{multicols}
\end{quote}

In the case of adverb-verb compounding, the auxiliary of the verb may change so
that it does not match the verb class' standard form. If the adverb ends in a
lengthened vowel, the first vowel of the auxiliary becomes lengthened if it is
not already. For example, the auxiliary for \textit{karlpii ngann} (``to
guzzle'') changes from the standard \textit{wann} to \textit{waann}.

\section{Nouns}\index{Nominal Derivation}

There are many noun derivational operations.

\subsection{Compounding}

Nominal compounds are bidirectional, meaning that the order of constituent nouns
does not change the meaning of the compound. In practical use, both orders are
used, with preference depending on the phonetics of the noun.

\begin{quote}
\begin{multicols}{2}
mountain \textit{gaypa}\\
stream \textit{munna}\\
$\Rightarrow$ mountain stream \textit{gaypamunna}

the Land \textit{Yawirra}\\
ground \textit{nnalu}\\
$\Rightarrow$ sacred place \textit{Yawirrannalu}
\end{multicols}
\end{quote}

\subsection{Collection}

Partial reduplication can be used to derive the collection of a noun. To derive
the collection, the first syllable is isolated, its coda removed, and added to
the front of the noun.

\begin{quote}
\begin{multicols}{2}
coconut \textit{wurna}\\
$\Rightarrow$ coconut tree \textit{wuwurna}

bone \textit{parrna}\\
$\Rightarrow$ corpse \textit{paparrna}
\end{multicols}
\end{quote}

\subsection{Container}

The container of a noun can be derived through the affix \textit{rna}.

\begin{quote}
\begin{multicols}{2}
arrow \textit{yungi}\\
$\Rightarrow$ quiver \textit{yungirna}

fruit \textit{yirli}\\
$\Rightarrow$ basket \textit{yirlinga}
\end{multicols}
\end{quote}
