\chapter{Semantics}

\section{Numbers}

The number system of Ngujari is built around a dual decimal-quinary system
involving the ten basic numerals which are outlined in the table below.

\begin{table}[h]
\centering
\begin{tabular}{lclc}
\textbf{numeral} & \textbf{word} & \textbf{numeral} & \textbf{word}\\
0 & nart & &\\
1 & naju & 6 & nalwi\\
2 & guu & 7 & puwa\\
3 & naa & 8 & tuja\\
4 & jaru & 9 & jawu\\
5 & yi & 10 & najuyi\\
\end{tabular}
\end{table}

Expressing numbers is simple for those under ten: the corresponding numeral is
used. Past ten, the base system begins to see use. Any numeral can be combined
with the words \textit{yi} (``five'') or \textit{najuyi} (``ten'') to multiply
that number. Large numbers are formed through multiples of five and ten as well
as extra numerals, which follow the multiples. There are two simple ways of
expressing any number, depending on the choice of five or ten as a base, but
bases can be combined in any number of ways.

\begin{quote}
\begin{multicols}{3}
twelve\\
\textit{najuyi guu}\\
\textit{guu-yi guu}

twenty-three\\
\textit{guu-najuyi naa}\\
\textit{jaru-yi naa}\\
\textit{najuyi guu-yi naa}

fifty\\
\textit{yi-najuyi}\\
\textit{najuyi-yi}\\
\end{multicols}
\end{quote}

When counting, a seperate tally system can optionally be used. ``Marks'', or
chosen multiples of five or ten, are expressed fully, but numbers in between are
expressed as the difference from the last mark. The following example shows a
speaker using this system:

\begin{sentence}
\textbf{guu-najuyi, naju, guu, naa, jaru, yi-yi, naju...}\\
\textit{twenty, twenty-one, twenty-two, twenty-three, twenty-four, twenty-five, twenty-six...}
\end{sentence}
