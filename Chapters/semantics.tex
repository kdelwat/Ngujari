\chapter{Semantics}

\section{Numbers}

The number system of Ngujari is built around a dual decimal-quinary system
involving the ten basic numerals which are outlined in the table below.

\begin{table}[h]
\centering
\begin{tabular}{lclc}
\textbf{numeral} & \textbf{word} & \textbf{numeral} & \textbf{word}\\
0 & nart & &\\
1 & naju & 6 & nalwi\\
2 & guu & 7 & puwa\\
3 & naa & 8 & tuja\\
4 & jaru & 9 & jawu\\
5 & yi & 10 & najuyi\\
\end{tabular}
\end{table}

Expressing numbers is simple for those under ten: the corresponding numeral is
used. Past ten, the base system begins to see use. Any numeral can be combined
with the words \textit{yi} (``five'') or \textit{najuyi} (``ten'') to multiply
that number. Large numbers are formed through multiples of five and ten as well
as extra numerals, which follow the multiples. There are two simple ways of
expressing any number, depending on the choice of five or ten as a base, but
bases can be combined in any number of ways.

\begin{quote}
\begin{multicols}{3}
twelve\\
\textit{najuyi guu}\\
\textit{guu-yi guu}

twenty-three\\
\textit{guu-najuyi naa}\\
\textit{jaru-yi naa}\\
\textit{najuyi guu-yi naa}

fifty\\
\textit{yi-najuyi}\\
\textit{najuyi-yi}\\
\end{multicols}
\end{quote}

When counting, a seperate tally system can optionally be used. ``Marks'', or
chosen multiples of five or ten, are expressed fully, but numbers in between are
expressed as the difference from the last mark. The following example shows a
speaker using this system:

\begin{sentence}
\textbf{guu-najuyi, naju, guu, naa, jaru, yi-yi, naju...}\\
\textit{twenty, twenty-one, twenty-two, twenty-three, twenty-four, twenty-five, twenty-six...}
\end{sentence}

\section{Colours}

In Ngujari, colours are derived from nouns through the suffix ``ku''. There are
six primary colours, detailed in the following table, along with their base noun.

\begin{table}[h]
\centering
\begin{tabular}{llll}
\textbf{colour} & \textbf{word} & \textbf{noun} & \textbf{meaning}\\
black & nguku & ngu & person\\
white & tumwaku & tumwa & sand\\
red & wirraku & wirra & blood\\
green & nurku & nurli & seaweed\\
yellow & puuki & puu & sun\\
\end{tabular}
\end{table}

Additional colours can be formed either through compounding or modifying a new
noun. All colours can be joined with others to form compounds.

\begin{quote}
\begin{multicols}{2}
black \textit{nguku}\\
white \textit{tumwaku}\\
$\Rightarrow$ grey \textit{nguku-tumwaku}

sunset \textit{kii}\\
$\Rightarrow$ orange \textit{kiiku}
\end{multicols}
\end{quote}

\section{Kinship}

The kinship system of Ngujari revolves around four \textit{totems}. The population is split into four:
\begin{itemize}
    \item \textit{bilru} (seal)
    \item \textit{gunya} (black wallaby)
    \item \textit{juunwi} (satin bowerbird)
    \item \textit{pilkiya} (platypus)
\end{itemize}

The pattern of totem inheritance is rigid but simple, with a woman of a specific
totem obliged to partner with a man of another prescribed totem to produce
offspring of an entirely different totem. However, the inheritance is ultimately
cyclic.

\begin{table}[h]
\centering
\begin{tabular}{lll}
\textbf{man} & \textbf{woman} & \textbf{offspring} \\
bilru      & gunya   & pilkiya  \\
gunya   & bilru      & juunwi \\
juunwi & pilkiya  & gunya   \\
pilkiya  & kuunwi & bilru\\
\end{tabular}
\end{table}

Closely related to the totem system is the \textit{nnurru}, a term for a sacred
place for all those members of a single totem. In contrast,
\textit{Yawirrannalu} (sacred place) is sacred for all Ngujari.

Many kinship terms are relative to totem, and are listed below:

\begin{table}[h]
\centering
\begin{tabular}{ll}
\textbf{word} & \textbf{relation}\\
wirrangu & parent, male or female\\
mulyi & mother\\
wiilyu & father\\
garanya & adult, of the same totem\\
jii & adult, of a different totem\\
giirki & offspring, baby (lit. seed)\\
\end{tabular}
\end{table}