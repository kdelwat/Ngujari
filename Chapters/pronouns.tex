\chapter{Pronouns}

\section{Personal}\index{Personal Pronouns}

Personal pronouns differ in three dimensions: person, plural, and gender. All
decline in the same way as regular nouns to indicate case. The following tables
list the pronouns:

\begin{table}[h]
\centering
\begin{tabular}{lccc}
 & \textbf{singular} & \textbf{dual} & \textbf{plural}\\
 \textbf{1st person} & jana & janna & juu\\
 \textbf{2nd person} & kurru & kunii & kurlu\\
 \textbf{3rd person} & nnarta & nnaja & nni\\
\end{tabular}
\caption{Child Personal Pronouns}
\end{table}

\begin{table}[h]
\centering
\begin{tabular}{lccc}
 & \textbf{singular} & \textbf{dual} & \textbf{plural}\\
 \textbf{1st person} & wa & ja & waya\\
 \textbf{2nd person} & ku & kuna & kuu\\
 \textbf{3rd person} & nna & nnara & nnaa\\
\end{tabular}
\caption{Adult Personal Pronouns}
\end{table}

\begin{table}[h]
\centering
\begin{tabular}{lccc}
 & \textbf{singular} & \textbf{dual} & \textbf{plural}\\
 \textbf{3rd person} & nnu & nnuka & nnunnu\\
\end{tabular}
\caption{Inanimate Personal Pronouns}
\end{table}

When speaking of a mob's elders, a personal pronoun is never used. The elder is
always referred to by their honorific title.

\section{Possessive}

Possessive pronouns are formed through a suffix placed on the relevant personal
pronoun, but only for the child and adult genders. For possession by elders, see
\ref{tribepos}. Inanimate objects cannot be possessive. For a child, the suffix
is \textit{ra} in first and second person and \textit{raa} in third person. For
an adult, the suffix is \textit{lu} for all persons.

\section{Interrogative}\index{Interrogative Pronouns}

The interrogative pronouns are strongly affected by case, particularly in the
case of location and time. The basic pronouns are detailed in the following
table:

\begin{table}[h]
\centering
\begin{tabular}{lc}
 \textbf{meaning} & \textbf{word}\\
 where & kiru\\
 when & tuu\\
 who, what & pii\\
 how & piima\\
 why & wiirtak\\
 how many & kirta\\
\end{tabular}
\caption{Interrogative Pronouns}
\end{table}

It is interesting to note that ``how'' is the same as ``what'' placed in the
instrumental case. The orientative and revertive cases can be applied to
\textit{kiru} (``where''), forming \textit{kirurni} (``whither/to where'') and
\textit{kirunga} (``whence/from where''), as well as to \textit{tuu} (``when''),
forming \textit{tuurni} (``to when'') and \textit{tuunga} (``from when'').

\section{Demonstrative}\index{Demonstrative Pronouns}

One set of demonstrative pronouns covers both proximal and distal objects.
Distinctions can be made in some cases between both gender and number. The
pronouns are found in the following table:

\begin{table}[h]
\centering
\begin{tabular}{lccc}
 \textbf{meaning} & \textbf{singular} & \textbf{dual} & \textbf{plural}\\
 there & naarla & naarla & naarla\\
 then & yaji & yaji & yaji\\
 that (animate) & yanna & yannara & yannaa\\
 that (inanimate) & yannu & yannuka & yannunnu\\
\end{tabular}
\caption{Demonstrative Pronouns}
\end{table}

Again, the pronouns \textit{naarla} and \textit{yaji} can assume the orientative
and revertive cases.

\section{Indefinite}\index{Indefinite Pronouns}

The regular indefinite pronouns are formed through modifying the interrogative
pronouns by appending the correct word, representing number. These words are
listed in the following table:

\begin{table}[h]
\centering
\begin{tabular}{lc}
 \textbf{number} & \textbf{word}\\
 none & nnayi\\
 singular & junga\\
 dual & marri\\
 plural & munaa\\
 all & nnaya\\
\end{tabular}
\caption{Indefinite Pronouns}
\end{table}

For example, ``everyone'' would be expressed as \textit{pii-nnaya} and ``some
two locations'' as \textit{kiru-marri}.
