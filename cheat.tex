\documentclass{article}
\usepackage[margin=2cm,landscape,a3paper]{geometry}
\usepackage{multicol}
\usepackage{multirow}
\usepackage{tipa}

\begin{document}
\begin{multicols*}{3}

\title{Ngujari Cheatsheet}
\author{Cadel Watson}
\date{\today}
\maketitle

\section{Phonology}

\subsection{Consonants}
\begin{tabular}{rcccccc}
  & \textbf{bilab.} & \textbf{alv.} & \textbf{p.-alv.} & \textbf{retro.} & \textbf{palatal} & \textbf{velar}\\
  \textbf{plosive} & \textipa{p} & \textipa{\|]{t}}\textit{(t)} & & \textipa{\|{]}{\textrtailt}}\textit{(rt)} & & \textipa{k}, \textipa{g}\\
  \textbf{nasal} & \textipa{m} & \textipa{\|]{n}}\textit{(n)} & \textipa{\textsubsquare{n}}\textit{(nn)} & \textipa{\|{]}{\textrtailn}}\textit{(rn)} & & \textipa{N}\textit{(ng)}\\
  \textbf{tap} & & \textipa{\|]R}\textit{(rr)} & & & &\\
  \textbf{fricat.} & & & \textipa{Z}\textit{(j)} & & &\\
  \textbf{approx.} & \textipa{w} & & & \textipa{\textturnrrtail}\textit{(r)} & \textipa{j}\textit{(y)} &\\
  \textbf{lat. approx} & & \textipa{\|]{l}}\textit{(l)} & & \textipa{\|]{\textrtaill}}\textit{(rl)} & &\\
\end{tabular}

\subsection{Vowels}
\begin{tabular}{rcc}
& \textbf{front} & \textbf{back}\\
\textbf{high} & \textipa{i}, \textipa{i:} & \textipa{u}, \textipa{u:}\\
\textbf{low} & \textipa{a}, \textipa{a:} &\\
\end{tabular}

\section{Nouns}

\subsection{Cases}
\begin{tabular}{lcc}
\textbf{case} & \textbf{abbreviation} & \textbf{suffix}\\
ergative & \textsc{erg} & -\\
nominative & \textsc{nom} & -j\\
accusative & \textsc{abs} & -l\\
instrumental & \textsc{ins} & -ma\\
comitative & \textsc{com} & -yi\\
orientative & \textsc{ori} & -rnu\\
revertive & \textsc{rev} & -rna\\
locative & \textsc{loc} & -rn\\
\end{tabular}

\section{Verbs}

\subsection{Classes}

\begin{tabular}{lccc}
\textbf{class} & \textbf{ending} & \textbf{auxiliary} & \textbf{negative particle}\\
\textbf{first} & -rr & k- & tu\\
\textbf{second} & -j & j- & ti\\
\textbf{third} & -nn & nn- & wuu\\
\end{tabular}

\subsection{Verb Conjugation}

\subsubsection{Gender of Subject}
\begin{tabular}{lcccc}
\textbf{class} & \textbf{child} & \textbf{adult} & \textbf{elder} & \textbf{inanimate}\\
first & uu & u & iiwa & a\\
second & awuu & awu & iwu & a\\
third & arruu & u & iwu & aa\\
\end{tabular}

\subsubsection{Person of Subject}
\begin{tabular}{lccc}
\textbf{class} & \textbf{1st} & \textbf{2nd} & \textbf{3rd} \\
first, second & -           & n          & m    \\
third & -           & ku          & mi    \\
\end{tabular}

\subsection{Auxiliary Conjugation}

\subsubsection{Tense}
\begin{tabular}{lcccc}
\textbf{class} & \textbf{remote past} & \textbf{past} & \textbf{present} & \textbf{future}\\
first & arlu & a & i & aa \\
second & arlu & a & i & aa\\
third & una & uma & uu & uuki\\
\end{tabular}

\subsubsection{Mood}
\begin{tabular}{lccccc}
\textbf{class} & \textbf{subj.} & \textbf{weak imp.} & \textbf{strong imp.} & \textbf{gnomic} & \textbf{dubitative}\\
first &  tiru & yii & ju & nga & tila\\
second & tirlu & yii & yuu & nga & ti\\
third &  tiru & yii &  ju & nga & ti\\
\end{tabular}

\subsection{Valence Modifiers}
\begin{tabular}{lllllll}
                                  &   & \multicolumn{5}{c}{\textbf{target}} \\
                                  &   & 0     & 1     & 2   & 3     & 4     \\
\multirow{5}{*}{\textbf{default}} & 0 & ---   & wi    & ji  & murnu & yurnu \\
                                  & 1 & wi    & ---   & naa & naki  & mu    \\
                                  & 2 & waa   & ka    & --- & naa   & naki  \\
                                  & 3 & wangu & waa   & ka  & ---   & naa   \\
                                  & 4 & wirru & wangu & waa & ka    & ---
\end{tabular}


\section{Adjectives}

Predicate adjectives are conjugated to gender:

\begin{tabular}{lcccc}
\textbf{class} & \textbf{child} & \textbf{adult} & \textbf{elder} & \textbf{inanimate}\\
\textbf{suffix} & uu & u & iiwa & a\\
\end{tabular}

\section{Pronouns}

\subsection{Personal}

\subsubsection{Child}
\begin{tabular}{lccc}
 & \textbf{singular} & \textbf{dual} & \textbf{plural}\\
 \textbf{1st person} & jana & janna & juu\\
 \textbf{2nd person} & kurru & kunii & kurlu\\
 \textbf{3rd person} & nnarta & nnaja & nni\\
\end{tabular}

\subsubsection{Adult}
\begin{tabular}{lccc}
 & \textbf{singular} & \textbf{dual} & \textbf{plural}\\
 \textbf{1st person} & wa & ja & waya\\
 \textbf{2nd person} & ku & kuna & kuu\\
 \textbf{3rd person} & nna & nnara & nnaa\\
\end{tabular}

\subsubsection{Inanimate}
\begin{tabular}{lccc}
 & \textbf{singular} & \textbf{dual} & \textbf{plural}\\
 \textbf{3rd person} & nnu & nnuka & nnunnu\\
\end{tabular}

\subsection{Possessive}

Possessive pronouns are formed through a suffix placed on the relevant personal
pronoun, but only for the child and adult genders. For possession by elders, see
\ref{tribepos}. Inanimate objects cannot be possessive.

\textbf{Child}: add \textit{ra} to relevant personal pronoun in first and second
person and \textit{raa} in third person.
\textbf{Adult}: add \textit{lu} to relevant personal pronoun.

\subsection{Interrogative}
\begin{tabular}{lc}
 \textbf{meaning} & \textbf{word}\\
 where & kiru\\
 when & tuu\\
 who, what & pii\\
 how & piima\\
 why & wiirtak\\
 how many & kirta\\
\end{tabular}

\subsection{Demonstrative}
\begin{tabular}{lccc}
 \textbf{meaning} & \textbf{singular} & \textbf{dual} & \textbf{plural}\\
 there & naarla & naarla & naarla\\
 then & yaji & yaji & yaji\\
 that (animate) & yanna & yannara & yannaa\\
 that (inanimate) & yannu & yannuka & yannunnu\\
\end{tabular}

\subsection{Indefinite}

Append the following to the relevant interrogative pronoun:

\begin{tabular}{lc}
 \textbf{number} & \textbf{word}\\
 none & nnayi\\
 singular & junga\\
 dual & marri\\
 plural & munaa\\
 all & nnaya\\
\end{tabular}

\section{Syntax}
\subsection{Alignment}

\textbf{All animate pronouns}: subject of transitive and intransitive verbs are
nominative, object is accusative.

\textbf{Other}: subject of transitive is ergative, subject of intransitive and
object of transitive are nominative.

\end{multicols*}
\end{document}
